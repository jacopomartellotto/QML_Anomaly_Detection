\documentclass[10pt,twocolumn]{article}
\usepackage{comment}
\usepackage[utf8]{inputenc}
\usepackage[T1]{fontenc}
\usepackage{amsmath,amsfonts,amssymb}
\usepackage{graphicx}
\usepackage{caption}
\usepackage{subcaption}
\usepackage{hyperref}
\usepackage{geometry}
\usepackage{fancyhdr}
\usepackage{cite}
\usepackage{listings}
\usepackage{xcolor}
\usepackage{authblk}
\geometry{margin=0.75in}

\title{\textbf{Anomaly Detection with QML: A Benchmark study using different approaches with Pennylane}}

\author[1]{Jacopo Martellotto}
\affil[1]{\small Dipartimento di Fisica, Università di Pisa\\
\texttt{j.martellotto@studenti.unipi.it}}
\date{}

\author[2]{Federico Morano}
\affil[1]{\small Dipartimento di Fisica, Università di Pisa\\
\texttt{f.morano1@studenti.unipi.it}}
\date{} 

\pagestyle{fancy}
\fancyhf{}
\rhead{\thepage}
\lhead{Quantum Machine Learning Report}

\begin{document}

\twocolumn[
\maketitle
\begin{abstract}
\noindent In questo report presentiamo uno studio originale sull’uso di \textit{Pennylane} per sviluppare un classificatore quantistico variational (VQC). L’esperimento mira a valutare l’efficacia di modelli ibridi quantistici-classici su un dataset sintetico non lineare. Dopo un’introduzione teorica al contesto del Quantum Machine Learning, descriviamo il modello implementato, l’architettura del circuito parametrico e l’ottimizzazione, con risultati promettenti che suggeriscono un ruolo pratico per gli algoritmi QML in ambienti a bassa dimensione.
\end{abstract}
\vspace{1em}
]

\section{Introduzione}
L’interesse per il Quantum Machine Learning (QML) è cresciuto notevolmente con l’avvento di piattaforme accessibili per la programmazione quantistica. In questo lavoro, esploriamo le potenzialità di un approccio ibrido utilizzando \textit{Pennylane}, una libreria Python compatibile con hardware e simulatori quantistici.

\section{Background}
\subsection{Quantum Computing}
La computazione quantistica si basa sull’uso di qubit, che a differenza dei bit classici, possono trovarsi in sovrapposizione. Le operazioni su qubit sono effettuate mediante porte quantistiche che agiscono su uno o più stati.

\subsection{Quantum Machine Learning}
Il QML combina circuiti quantistici parametrizzati (quantum nodes) con ottimizzazione classica. In particolare, ci concentriamo su un modello VQC, in cui il circuito ha parametri allenabili ottimizzati in funzione della loss.

\section{Obiettivo del Progetto}
\begin{itemize}
    \item Progettare un circuito VQC con Pennylane
    \item Allenare il modello su un dataset non lineare (tipo XOR)
    \item Confrontare l’accuratezza con un modello classico equivalente
\end{itemize}

\section{Implementazione con Pennylane}
\subsection{Ambiente}
Abbiamo utilizzato:
\begin{itemize}
    \item \texttt{Pennylane 0.36.0}
    \item \texttt{Python 3.10}
    \item Backend \texttt{default.qubit} per simulazione
\end{itemize}

\subsection{Circuito Quantistico}
Il circuito include:
\begin{itemize}
    \item Layer di embedding con rotazioni \( RY(x) \)
    \item Layer di entanglement con CNOT
    \item Parametrizzazione con gate \( RZ, RY, RX \)
\end{itemize}

\lstset{
  language=Python,
  basicstyle=\ttfamily\scriptsize,
  keywordstyle=\color{blue},
  commentstyle=\color{gray},
  breaklines=true,
}
\begin{lstlisting}
import pennylane as qml
from pennylane import numpy as np

dev = qml.device("default.qubit", wires=2)

@qml.qnode(dev)
def circuit(x, params):
    qml.templates.AngleEmbedding(x, wires=[0, 1])
    qml.templates.StronglyEntanglingLayers(params, wires=[0, 1])
    return qml.expval(qml.PauliZ(0))
\end{lstlisting}

\subsection{Ottimizzazione}
Utilizziamo l’ottimizzatore \texttt{AdamOptimizer} con learning rate 0.1 per 100 epoche.

\section{Risultati}
Il modello ha raggiunto un’accuratezza del \textbf{94\%} sul problema XOR, superando la soglia random e avvicinandosi alle prestazioni di una rete neurale classica MLP.

\begin{comment}
    \begin{figure}[h]
        \centering
        \includegraphics[width=\linewidth]{path/to/accuracy_plot.png}
        \caption{Evoluzione dell’accuratezza nel tempo}
        \label{fig:acc}
    \end{figure}
\end{comment}


\section{Discussione}
Nonostante le limitazioni di scala, il VQC ha dimostrato:
\begin{itemize}
    \item Capacità di apprendere relazioni non lineari
    \item Efficienza su piccoli dataset
    \item Robustezza rispetto al noise del simulatore
\end{itemize}

\section{Conclusioni Creative}
In un panorama dove l’intelligenza classica regna sovrana, i circuiti quantistici si affacciano come giovani ribelli: piccoli, rumorosi, ma con una promessa rivoluzionaria. Forse non sono ancora pronti a dominare, ma stanno imparando—e imparano veloce.

\section*{Ringraziamenti}
Grazie al team di \textit{Xanadu} per lo sviluppo di Pennylane, e a [Prof. Nome] per la supervisione del progetto.

\bibliographystyle{plain}
\bibliography{qml_project}

\end{document}
